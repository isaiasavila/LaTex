\documentclass[a4paper,12pt]{article}
\usepackage[top=2cm,bottom=2cm,left=2cm,right=2cm]{geometry}
\usepackage[utf8]{inputenc}
\usepackage{amsmath,amsfonts,amssymb,verbatim}
\DeclareMathOperator{\sen}{sen}
\DeclareMathOperator{\cossec}{cossec}
\DeclareMathOperator{\tg}{tg}

\begin{document}

\begin{enumerate}
\begin{comment}
Comentário, muito útil para 
\end{comment}
\item Seja a função $f: \mathbb{R} \to \mapsto \mathbb{R}$ definida por $f(x)= \dfrac{1}{2}x^2 - 2x + 1$

\item Seja a função $$f: \mathbb{R} \to \mathbb{R}$$
 $$x \mapsto \dfrac{1}{2}x^2 - 2x + 1$$
\item Seja a função $$f: \mathbb{R} \to \mathbb{R}$$
$$f(x)=
\begin{cases}
	x^2 - 1; \, \textrm{se } x \geq 1 \\
	x - 3; \, \textrm{se } -1 \leq x<1 \\
	2x+1; \, \textrm{se } x < -1
\end{cases}
$$
\item Seja a função $$f: \mathbb{R} \to \mathbb{R}$$
$$f(x)=2^{x-1}$$
\item Seja a função $$f: \mathbb{R}^*_+ \to \mathbb{R}$$
$$f(x)= \log_2 x$$
$$f(x)= \ln x$$
$$f(x)= \cos x$$
$$f(x)= \tan x$$
\begin{comment}
Exemplo da notação original de seno
\end{comment}
$$f(x)= \sin x$$
\begin{comment}
Exemplo da notação de seno com o jeitinho brasileiro
\end{comment}
$$f(x)= \sen \left(x - \frac{\pi}{2} \right)$$
$$f(x)= \tg \left\{x - \frac{\pi}{4} \right\}$$
$$f(x)= \cossec \left[x - \frac{\pi}{8} \right]$$
	\begin{enumerate}
		\item Esboce o gráfico da função.
	\end{enumerate}
\end{enumerate}

\begin{enumerate}
\item Sejam os conjuntos $A =\{1;\, 2;\; 3;\ 4\}$,
$B = \{x \in \mathbb{Z}\,|\,-2 \leq x < 4\}$ e $C = \{x \in \mathbb{N}\,|\,x \geq 2\}$.
Responda aos itens abaixo:


\begin{enumerate}
\item $A \cap B$
\item $B \cup C$
\item $A - C$
\item $C \setminus B$
\end{enumerate}
\item Classifique em verdadeiro ou falso.
\begin{enumerate}
\item $\mathbb{N} \subset \mathbb{Z}$
\item $\mathbb{R} \supset \mathbb{Q}$
\item $\mathbb{N} \not\subset \mathbb{Z}$
\item $\mathbb{R} \not\supset \mathbb{Q}$
\item $0 \not\in \mathbb{I} \setminus \mathbb{Q}$
\item $\forall x \in \mathbb{N}$, temos $x \geq 0$
\item $\exists x \in \mathbb{R} \,|\, \sqrt{x} \not\in \mathbb{R}$.
\item $7 \not\in \{x\in \mathbb{N} \,|\, x\, \textrm{ é par}$
\item $-5 \in \mathbb{R}^*_+$
\item $0 \in \varnothing$
\end{enumerate}
\end{enumerate}

\begin{enumerate}
\item Soma
$a + b$
\item Subtração
$a - b$
\item Multiplicação
$a \cdot b$ ou $a \times b$
\item Divisão
$a \div b$
\item Fração
$\frac{a}{b}$ ou $\dfrac{a}{b}$
\item Raiz Quadrada
$\sqrt[3]{a}$ ou $\sqrt[8]{a}$ ou $\sqrt{a}$
\item Potenciação
$a^b$ ou $a^{b+c}$ ou $a^{10}$ ou $a^\frac{1}{2}$
\item Itens
$a_1$,$a_10$,$a_{10}$
\end{enumerate}

\begin{center}
\textbf{Equa\c c\~ao polinomial do 2$^\circ$ grau.}
\end{center}
\begin{flushleft}
\textit{Equação polinomial do 2º grau.}
\end{flushleft}
\begin{flushright}
\underline{Equação polinomial do 2º grau.}
\end{flushright}

\textbf{\textit{\underline{Uma equação da forma}} $$a^2 + bx + c = 0$$, com $a \neq 0$ será chamada de} equação polinomial do 2º grau.

A solução dessa equação é dada por
$$x = \frac{-b \pm \sqrt{b^2 - 4ac}}{2a}$$

\begin{enumerate}
\item Vov\' o 
\begin{enumerate}
\item Gustavo
\item João
\end{enumerate}
\item Vov\^ o.
\begin{itemize}
\item Edelvira
\item Julia
\end{itemize}
\end{enumerate}
\end{document}