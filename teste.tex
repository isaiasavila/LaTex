\documentclass[a4paper,12pt]{article}
\usepackage[top=2cm,bottom=2cm,left=2cm,right=2cm]{geometry}
\usepackage[utf8]{inputenc}
\usepackage{amsmath,amsfonts,amssymb,verbatim}
\usepackage{graphicx}
\usepackage{float}
\usepackage[portuguese]{babel}
\DeclareMathOperator{\sen}{sen}
\DeclareMathOperator{\cossec}{cossec}
\DeclareMathOperator{\tg}{tg}
\newcommand{\limite}{\displaystyle\lim}
\newcommand{\integral}{\displaystyle\int}
\newcommand{\somatorio}{\displaystyle\sum}
\begin{document}

\begin{center}
\begin{tabular}{lp{.8\textwidth}}
\hline
	\begin{tabular}{c}\includegraphics[scale=.25]{Nova Imagem de Bitmap.png}\end{tabular}&
	\begin{tabular}{r}
		Circunferência \\ 
		isaiasavilasantos@gmail.com.br \\
		Professor Isaías Ávila
	\end{tabular}

\end{tabular}
\end{center}

\begin{enumerate}
\item A tabela \ref{tabela-der} abaixo representa as derivadas básicas:

\centering
\begin{tabular}{|l|r|}
	\hline
	Função & Derivada \\ \hline
	$f(x)=x^n$ & $f'(x)=nx^{n-1}$ \\ \hline
	$f(x)=Log_x a$ & $f'(x)=\dfrac{1}{(\ln a)x}$ \\ \hline
	
\end{tabular}

\begin{table}[!htb]
\begin{center}
\begin{tabular}{||c||p{10cm}||}
	\hline \hline
	Função & Derivada \\ \hline \hline
	$f(x)=x^n$ & $f'(x)=nx^{n-1}$ \\ \hline \hline
	$f(x)=Log_x a$ & $f'(x)=\dfrac{1}{(\ln a)x}$ \\ \hline \hline
	
\end{tabular}
\caption{Tabela básica de derivadas.}
\label{tabela-der}
\end{center}
\end{table}

\end{enumerate}

\begin{enumerate}
\item Calcule o valor da figura \ref{meu-rotulo}.
\begin{figure}[!htb]
\includegraphics[scale=.25]{Nova Imagem de Bitmap.png}
\caption{Questão sobre circunferência.}
\label{meu-rotulo}
\end{figure}

\item Calcule o valor da figura \ref{meu-outro-rotulo}.
\begin{figure}[H]
\centering
\includegraphics[scale=.15]{Nova Imagem de Bitmap.png}
\caption{Questão sobre circunferência.}
\label{meu-outro-rotulo}
\end{figure}
\end{enumerate}

\begin{enumerate}
\begin{comment}
Tags para cálculo
\end{comment}
\item Calcule os limites abaixo.
\begin{enumerate}
\item $\limite_{x \to 1} \dfrac{x^2 - 1}{x-1}$
\item $\displaystyle\lim_{x \to 1} \frac{x^2 - 1}{x-1}$
\end{enumerate}
\item Seja a função definida por $f(x)=x^2- \sqrt{x}$. Calcule as derivadas abaixo.
\begin{enumerate}
\item $f'$
\item $f''$
\item $f'''$
\item $f^{(iv)}$
\item $f^{(v)}$
\item $\dfrac{df}{dx}$
\item $\dfrac{d^2f}{dx^2}$
\item $\dfrac{d^3f}{dx^3}$
\end{enumerate}
\item Seja a função definida por $f(x,\,y)=yx^2- \sqrt{x} + y^3$. Calcule as derivadas abaixo.
\begin{enumerate}
\item $\dfrac{\partial f}{\partial x}$
\item $\dfrac{\partial^2f}{\partial x^2}$
\item $\dfrac{\partial^3f}{\partial x^3}$
\item $\dfrac{\partial }{\partial x}\left(\dfrac{\partial f}{\partial y}\right)$
\end{enumerate}

\item Calcule as integrais abaixo
\begin{enumerate}
\item $\int_1^5 x^2\cos x\,dx$
\item $$\int_1^7 x^2\cos x\,dx$$
\item $\displaystyle\int_2^4 x^3\cos x\,dx$
\item $\integral_0^4 x^2\cos x\,dx$
\end{enumerate}

\item Somatório
\begin{enumerate}
\item $\sum_{i=1}^n a_i$
\item $$\sum_{i=1}^n a_i$$
\item $\displaystyle\sum_{i=1}^n a_i$
\item $\somatorio_{i=1}^n a_i$
\end{enumerate}
\end{enumerate}

\begin{enumerate}
\begin{comment}
Tags para geometria analítica
\end{comment}
\item Seja o segmento $\overline{AB}$. A partir dele podemos definir os segmentos orientados $\overrightarrow{AB}$
e $\overrightarrow{BA}$. Seja $\vec{AB}$ e $\vec{u}$.
\item Sejam os vetores $\vec{u} = (1;\,-1;\,2)$ e $\vec{u} = (2;\,5;\,-1)$. Calcule o seguinte.
\begin{enumerate}
\item $\vec{u} \cdot \vec{v}$
\item $\langle \vec{u} \cdot \vec{v} \rangle$
\item $\vec{u} \times \vec{v}$
\item $|\vec{u}|$ $\|\vec{u}\|$
\item $\left\|\overrightarrow{AB}\right\|$
\end{enumerate}
\begin{enumerate}
\item Verifique se $\vec{u} \perp \vec{v}$.
\item Sejam os planos $\alpha : x-2y+6z-3=0$ $\beta : x-2y+6z-3=0$.
\item Sejam os vetores $\vec{u}=(x_0;\,y_0;\,z_0)$ e $\vec{v}=(x_1;\,y_1;\,z_1)$. Temos que:
$$
\vec{u} \times \vec{v} =
\begin{vmatrix}
\vec{i} & \vec{j} & \vec{k}\\
x_0 & y_0 & z_0\\
x_1 & y_1 & z_1
\end{vmatrix}
$$
\end{enumerate}
\end{enumerate}

\begin{enumerate}
\item Considere a matriz
$\begin{bmatrix}
 1 & 10 & -5 \\
 6 & 7 & 8
 \end{bmatrix}
$
\item Determine $x$, $y$ e $z$ na equação
\item Considere a matriz $m\times n$ dada por
$\begin{bmatrix}
1 & -2 & 4  \\
6 & 8 & 10  \\
3 & 7 & 13
 \end{bmatrix}
 =
  \begin{bmatrix}
 x\\
 y\\
 z
 \end{bmatrix}
 \begin{bmatrix}
 2\\
 10\\
 6
 \end{bmatrix}
$
\item Calcule o determinante
$\begin{vmatrix}
 1 & 10 & -5 \\
 6 & 7 & 8 \\
 7 & 45 & 12
 \end{vmatrix}
$
\item Considere a matriz
$M = \begin{bmatrix}
 1 & 10 & -5 \\
 6 & 7 & 8
 \end{bmatrix}
$. Calcule o que for solicitado abaixo.
\begin{enumerate}
\item $\det M$
\item $M^{-1}$
\item $M^T$
\end{enumerate}
\end{enumerate}

\begin{enumerate}
\item Seja a função $f: \mathbb{R} \to \mapsto \mathbb{R}$ definida por $f(x)= \dfrac{1}{2}x^2 - 2x + 1$

\item Seja a função $$f: \mathbb{R} \to \mathbb{R}$$
 $$x \mapsto \dfrac{1}{2}x^2 - 2x + 1$$
\item Seja a função $$f: \mathbb{R} \to \mathbb{R}$$
$$f(x)=
\begin{cases}
	x^2 - 1; \, \textrm{se } x \geq 1 \\
	x - 3; \, \textrm{se } -1 \leq x<1 \\
	2x+1; \, \textrm{se } x < -1
\end{cases}
$$
\item Seja a função $$f: \mathbb{R} \to \mathbb{R}$$
$$f(x)=2^{x-1}$$
\item Seja a função $$f: \mathbb{R}^*_+ \to \mathbb{R}$$
$$f(x)= \log_2 x$$
$$f(x)= \ln x$$
$$f(x)= \cos x$$
$$f(x)= \tan x$$
\begin{comment}
Exemplo da notação original de seno
\end{comment}
$$f(x)= \sin x$$
\begin{comment}
Exemplo da notação de seno com o jeitinho brasileiro
\end{comment}
$$f(x)= \sen \left(x - \frac{\pi}{2} \right)$$
$$f(x)= \tg \left\{x - \frac{\pi}{4} \right\}$$
$$f(x)= \cossec \left[x - \frac{\pi}{8} \right]$$
	\begin{enumerate}
		\item Esboce o gráfico da função.
	\end{enumerate}
\end{enumerate}

\begin{enumerate}
\item Sejam os conjuntos $A =\{1;\, 2;\; 3;\ 4\}$,
$B = \{x \in \mathbb{Z}\,|\,-2 \leq x < 4\}$ e $C = \{x \in \mathbb{N}\,|\,x \geq 2\}$.
Responda aos itens abaixo:


\begin{enumerate}
\item $A \cap B$
\item $B \cup C$
\item $A - C$
\item $C \setminus B$
\end{enumerate}
\item Classifique em verdadeiro ou falso.
\begin{enumerate}
\item $\mathbb{N} \subset \mathbb{Z}$
\item $\mathbb{R} \supset \mathbb{Q}$
\item $\mathbb{N} \not\subset \mathbb{Z}$
\item $\mathbb{R} \not\supset \mathbb{Q}$
\item $0 \not\in \mathbb{I} \setminus \mathbb{Q}$
\item $\forall x \in \mathbb{N}$, temos $x \geq 0$
\item $\exists x \in \mathbb{R} \,|\, \sqrt{x} \not\in \mathbb{R}$.
\item $7 \not\in \{x\in \mathbb{N} \,|\, x\, \textrm{ é par}$
\item $-5 \in \mathbb{R}^*_+$
\item $0 \in \varnothing$
\end{enumerate}
\end{enumerate}

\begin{enumerate}
\item Soma
$a + b$
\item Subtração
$a - b$
\item Multiplicação
$a \cdot b$ ou $a \times b$
\item Divisão
$a \div b$
\item Fração
$\frac{a}{b}$ ou $\dfrac{a}{b}$
\item Raiz Quadrada
$\sqrt[3]{a}$ ou $\sqrt[8]{a}$ ou $\sqrt{a}$
\item Potenciação
$a^b$ ou $a^{b+c}$ ou $a^{10}$ ou $a^\frac{1}{2}$
\item Itens
$a_1$,$a_10$,$a_{10}$
\end{enumerate}

\begin{center}
\textbf{Equa\c c\~ao polinomial do 2$^\circ$ grau.}
\end{center}
\begin{flushleft}
\textit{Equação polinomial do 2º grau.}
\end{flushleft}
\begin{flushright}
\underline{Equação polinomial do 2º grau.}
\end{flushright}

\textbf{\textit{\underline{Uma equação da forma}} $$a^2 + bx + c = 0$$, com $a \neq 0$ será chamada de} equação polinomial do 2º grau.

A solução dessa equação é dada por
$$x = \frac{-b \pm \sqrt{b^2 - 4ac}}{2a}$$

\begin{enumerate}
\item Vov\^ o 
\begin{enumerate}
\item Gustavo
\item João
\end{enumerate}
\item Vov\' o.
\begin{itemize}
\item Edelvira
\item Julia
\end{itemize}
\end{enumerate}
\end{document}